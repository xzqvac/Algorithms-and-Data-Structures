\documentclass{article}

\usepackage{amsmath}
\usepackage{amssymb}
\usepackage{enumitem}
\usepackage[polish]{babel}
\usepackage[T1]{fontenc}
\usepackage[utf8]{inputenc}
\usepackage[margin=3cm]{geometry}
\usepackage[edges]{forest}
\usepackage{listings}
\usepackage{multicol}
\usepackage[utf8]{inputenc}
\usepackage[breaklinks]{hyperref}

\title{
    \Huge\textbf{Algorytmy i struktury danych} \\
    \vspace{2mm}
    \huge{Sortowanie i kopce} 
    \date{}
}

\begin{document}
    \maketitle
    \boldmath
    \section*{Przygotowanie do kolokwium}
    Przyjmując, że \verb+tabA[] = {1, 2, 3, 4, 5, 6, 7}+ oraz \verb+tabB[] = {7, 6, 5, 4, 3, 2, 1}+ i stosując algorytmy
    sortujące ściśle według procedur z pliku sort2023.cc wykonaj polecenia:

    \subsection*{Zadanie 1}
    Ile dokładnie porównań (między elementami tablic) wykona \verb+insertion_sort(tabB)+, a ile \verb+insertion_sort(tabA)+?

    \begin{center}
        \textbf{Przypadek optymistyczny}
        \begin{center}
            Dla tablicy posortowanej rosnąco o długości n, \verb+insertion_sort+ wykona \textbf{n-1 porównań}. \\
            \verb+insertion_sort+ dla tablicy tabA wykona \textbf{6 porównań}. \\ 
            Złożoność czasowa: $O(n)$
        \end{center}

        \textbf{Przypadek pesymistyczny}
        \begin{center}
            Dla tablicy posortowanej malejąco o długości n, \verb+insertion_sort+ wykona $\frac{n^2 - n}{2}$ \textbf{porównań}. \\ 
            \verb+insertion_sort+ dla tablicy tabA wykona \textbf{21 porównań}. \\
            Złożoność czasowa: $O(n^2)$
        \end{center}
    \end{center}
        
\end{document}