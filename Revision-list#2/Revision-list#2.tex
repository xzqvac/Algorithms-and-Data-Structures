\documentclass{article}

\usepackage{amsmath}
\usepackage{amssymb}
\usepackage{enumitem}
\usepackage[polish]{babel}
\usepackage[T1]{fontenc}
\usepackage[utf8]{inputenc}
\usepackage[margin=3cm]{geometry}
\usepackage[edges]{forest}
\usepackage{listings}
\usepackage{multicol}
\usepackage[utf8]{inputenc}
\usepackage[breaklinks]{hyperref}
\usepackage{listings}
\usepackage{xcolor}
\lstset { %
    language=C++,
    backgroundcolor=\color{black!5}, % set backgroundcolor
    basicstyle=\footnotesize,% basic font setting
}

\title {
    \Huge\textbf{Algorytmy i struktury danych} \\
    \vspace{2mm}
    \huge{Lista zadań 7 - B-drzewa} 
    \date{}
}

\begin{document}
    \maketitle
    \boldmath
    \section*{Zadanie 1}
        Zapisz warunki jakie muszą spełniać klucze drzewa BST.
        \begin{itemize}
            \item Klucze w lewym poddrzewie są mniejsze od klucza w danym węźle.
            \item Klucze w prawym poddrzewie są większe lub równe kluczowi w danym węźle.
        \end{itemize}
    
\end{document}