\documentclass{article}

\usepackage{amsmath}
\usepackage{amssymb}
\usepackage{enumitem}
\usepackage[polish]{babel}
\usepackage[T1]{fontenc}
\usepackage[utf8]{inputenc}
\usepackage[margin=3cm]{geometry}
\usepackage[edges]{forest}
\usepackage{listings}
\usepackage{multicol}
\usepackage[utf8]{inputenc}
\usepackage[breaklinks]{hyperref}
\usepackage{listings}
\usepackage{xcolor}
\lstset { %
    language=C++,
    backgroundcolor=\color{black!5}, % set backgroundcolor
    basicstyle=\footnotesize,% basic font setting
}

\title {
    \Huge\textbf{Algorytmy i struktury danych} \\
    \vspace{2mm}
    \huge{Lista zadań 7 - B-drzewa} 
    \date{}
}

\begin{document}
    \maketitle
    \boldmath
    \section*{Zadanie 1}
        Jakie informacje przechowujemy w węźle B-drzewa? Podaj definicję
        B-drzewa

        \begin{lstlisting}
            struct BTree
            {
                bool isLeaf;
                size_t n;
                int32_t *keys;
                BTree **children;
            };
        \end{lstlisting}

        W węźle B-drzewa możemy przechowywać kilka wartości posortowane niemalejąco.
        W B-drzewie każda wartość ma wskaźnik na lewe i prawe dziecko. Lewe dziecko
        zawiera wartości mniejsze od wartości rodzica, prawe dziecko zawiera wartości
        większe bądź równe od wartości rodzica. Każdy węzeł za wyjątkiem korzenia 
        będzie miał o jedno dziecko więcej niż ma kluczy.
        \begin{enumerate}
            \item Każdy węzeł posiada n kluczy posortowanynch niemalejąco
            \item Każdy węzeł posiada n + 1 wskaźników na swoje dzieci
            \item Każdy węzeł posiada minimalnie T - 1 kluczy(nie dotyczy korzenia)
            \item Każdy węzeł może mieć maksymalnie 2T - 1 kluczy
            \item Wszystkie liście są na tym samym poziomie
            \item Dziecko pomiędz kluczem k1 a kluczem k2, zawiera klucze większe
            od k1 i mniejsze od k2
        \end{enumerate} 
    
\end{document}